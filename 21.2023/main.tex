\let\negmedspace\undefined
\let\negthickspace\undefined
\documentclass[journal,12pt,twocolumn]{IEEEtran}
\usepackage{cite}
\usepackage{amsmath,amssymb,amsfonts,amsthm}
\usepackage{algorithmic}
\usepackage{graphicx}
\usepackage{textcomp}
\usepackage{xcolor}
\usepackage{txfonts}
\usepackage{listings}
\usepackage{enumitem}
\usepackage{mathtools}
\usepackage{gensymb}
\usepackage{comment}
\usepackage[breaklinks=true]{hyperref}
\usepackage{tkz-euclide} 
\usepackage{listings}
\usepackage{gvv}                                        
\def\inputGnumericTable{}                                 
\usepackage[latin1]{inputenc}                                
\usepackage{color}                                            
\usepackage{array}                                            
\usepackage{longtable}                                       
\usepackage{calc}                                             
\usepackage{multirow}                                         
\usepackage{hhline}                                           
\usepackage{ifthen}                                           
\usepackage{lscape}

\newtheorem{theorem}{Theorem}[section]
\newtheorem{problem}{Problem}
\newtheorem{proposition}{Proposition}[section]
\newtheorem{lemma}{Lemma}[section]
\newtheorem{corollary}[theorem]{Corollary}
\newtheorem{example}{Example}[section]
\newtheorem{definition}[problem]{Definition}
\newcommand{\BEQA}{\begin{eqnarray}}
\newcommand{\EEQA}{\end{eqnarray}}
\newcommand{\define}{\stackrel{\triangle}{=}}
\theoremstyle{remark}
\newtheorem{rem}{Remark}
\begin{document}

\bibliographystyle{IEEEtran}
\vspace{3cm}

\title{Probability Assignment}
\author{EE22BTECH11022-G.SAI HARSHITH$^{*}$% <-this % stops a space
}
\maketitle
\newpage
\bigskip
\renewcommand{\thefigure}{\theenumi}
\renewcommand{\thetable}{\theenumi}

Question: Let $\{X_n\}_{n \geq 1}$ and Let $\{Y_n\}_{n \geq 1}$ be two sequences of random variables and $X$ and $Y$
be two random variables, all of them defined on the same probability space.
Which one of the following statements is true?
\begin{enumerate}[label=(\Alph*)]
\item If $\{X_n\}_{n \geq 1}$ converges in distribution to a real constant $c$, then $\{X_n\}_{n \geq 1}$
converges in probability to $c$.
\item If $\{X_n\}_{n \geq 1}$ converges in probability to $X$, then $\{X_n\}_{n \geq 1}$ converges in $3^{rd}$ mean
to $X$.
\item If $\{X_n\}_{n \geq 1}$ converges in distribution to $X$ and $\{Y_n\}_{n \geq 1}$ converges in
distribution to $Y$, then $\{X_n + Y_n\}_{n \geq 1}$ converges in distribution to $X+Y$.
\item If $\{E\brak{X_n}\}_{n \geq 1}$ converges to $E(X)$, then $\{X_n\}_{n \geq 1}$ converges in $1^{st}$ mean to $X$.
\end{enumerate}
\solution \\
$X_n$ converges in distribution to $X$, $X_n \xrightarrow{d} X$, then for all x,
\begin{align}
F_{X_n}(x) &\xrightarrow{} F_X(x)
\end{align}
$X_n$ converges in probability to $X$, $X_n \xrightarrow{p} X$, then for all $\epsilon > 0$,
\begin{align}
lim_{n \to \infty} \pr{|X_n-X|>\epsilon}&=0
\label{eq:1}
\end{align}
For $\epsilon > 0$, $B$ be defined as
\begin{align}
B&=\{x: |x-c| \geq \epsilon\}
\end{align}
Now,
\begin{align}
\pr{|X_n-c| \geq \epsilon}&= \pr{X_n \in B}
\end{align}
Using Portmanteau Lemma, if $X_n \xrightarrow{d} c$, we have
\begin{align}
\limsup\limits_{n \to \infty}\pr{X_n \in B} & \leq \pr{c \in B}\\
& \leq \pr{|0-0|\geq \epsilon}\\
& \leq \pr{0\geq \epsilon}\\
& \leq 0\\
&=0\\
lim_{n \to \infty} \pr{|X_n-c|>\epsilon}&=0
\end{align}
From \eqref{eq:1}, $X_n \xrightarrow{p} c$. So, we have
\begin{align}
X_n \xrightarrow{d} c \implies X_n \xrightarrow{p} c
\end{align}
Option (A) is correct.
\end{document}
